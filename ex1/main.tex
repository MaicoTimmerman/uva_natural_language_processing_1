\documentclass[a4paper]{article}
\addtolength{\hoffset}{-2.25cm}
\addtolength{\textwidth}{4.5cm}
\addtolength{\voffset}{-3.25cm}
\addtolength{\textheight}{5cm}
\setlength{\parskip}{0pt}
\setlength{\parindent}{0in}

%----------------------------------------------------------------------------------------
%	PACKAGES AND OTHER DOCUMENT CONFIGURATIONS
%----------------------------------------------------------------------------------------

\usepackage{blindtext} % Package to generate dummy text
\usepackage[utf8]{inputenc} % Use UTF-8 encoding
\usepackage{microtype} % Slightly tweak font spacing for aesthetics
\usepackage[english]{babel} % Language hyphenation and typographical rules
\usepackage{amsthm, amsmath, amssymb} % Mathematical typesetting
\usepackage{mathtools} % Mathematical typesetting
\usepackage{float} % Improved interface for floating objects
\usepackage[final, colorlinks = true,
            linkcolor = black,
            citecolor = black]{hyperref} % For hyperlinks in the PDF
\usepackage{graphicx, multicol} % Enhanced support for graphics
\usepackage[dvipsnames]{xcolor} % Driver-independent color extensions
\usepackage{marvosym, wasysym} % More symbols
\usepackage{rotating} % Rotation tools
\usepackage{censor} % Facilities for controlling restricted text
\usepackage{listings, style/lstlisting} % Environment for non-formatted code, !uses style file!
\usepackage{pseudocode} % Environment for specifying algorithms in a natural way
\usepackage{style/avm} % Environment for f-structures, !uses style file!
\usepackage{booktabs} % Enhances quality of tables
\usepackage{bm} % Enhances quality of tables
\usepackage{tikz-qtree} % Easy tree drawing tool
\usepackage{enumitem} % enumerate with custom labels
\tikzset{every tree node/.style={align=center,anchor=north},
         level distance=2cm} % Configuration for q-trees
\usepackage{style/btree} % Configuration for b-trees and b+-trees, !uses style file!
\usepackage[backend=biber,style=numeric,
            sorting=nyt]{biblatex} % Complete reimplementation of bibliographic facilities
\addbibresource{ecl.bib}
\usepackage{csquotes} % Context sensitive quotation facilities
\usepackage[yyyymmdd]{datetime} % Uses YEAR-MONTH-DAY format for dates
\renewcommand{\dateseparator}{-} % Sets dateseparator to '-'
\usepackage{fancyhdr} % Headers and footers
\pagestyle{fancy} % All pages have headers and footers
\fancyhead{}\renewcommand{\headrulewidth}{0pt} % Blank out the default header
\fancyfoot[L]{} % Custom footer text
\fancyfoot[C]{} % Custom footer text
\fancyfoot[R]{\thepage} % Custom footer text
\newcommand{\note}[1]{\marginpar{\scriptsize \textcolor{red}{#1}}} % Enables comments in red on margin

%----------------------------------------------------------------------------------------

\newcommand*{\vertbar}{\rule[-1ex]{0.5pt}{2.5ex}}
\newcommand*{\horzbar}{\rule[.5ex]{2.5ex}{0.5pt}}

\newcommand{\E}[1]{{\mathbb{E}\left[#1\right]}}
\newcommand{\var}[1]{{\text{var}\left[#1\right]}}
\newcommand{\diffpart}[1]{\frac{\partial}{\partial #1}}
\newcommand{\diffpartt}[2]{\frac{\partial#1}{\partial #2}}
\newcommand{\?}{\stackrel{?}{=}}
\newcommand{\intinf}{\int\limits_{-\infty}^{\infty}}
\newcommand{\intnulinf}{\int\limits_{0}^{\infty}}
\newcommand{\intpi}{\int\limits_{0}^{2\pi}}
\newcommand{\sume}[1]{\sum\limits_{#1}}
\newcommand{\sumt}[2]{\sum\limits_{#1}^{#2}}
\newcommand{\prodt}[2]{\prod\limits_{#1}^{#2}}
\newcommand{\argmin}[1]{\underset{#1}{\mathop{\mathrm{argmin}}}}
\newcommand{\argmax}[1]{\underset{#1}{\mathop{\mathrm{argmax}}}}
\newcommand{\normdist}[1]{\mathcal{N}(#1)}
\newcommand{\half}{\frac{1}{2}}
\newcommand{\ind}[2]{\mathbb{I}(#1=#2)}
\renewcommand{\arraystretch}{1.3}

% VECTOR NOTATION
\newcommand{\mSig}{\bm{\Sigma}}
\newcommand{\mSigi}{\bm{\Sigma}^{-1}}
\newcommand{\mS}{\bm{S}}
\newcommand{\mX}{\mathbf{X}}
\newcommand{\mM}{\mathbf{M}}
\newcommand{\mV}{\mathbf{V}}
\newcommand{\mY}{\mathbf{Y}}
\newcommand{\mSi}{\bm{S}^{-1}}
\newcommand{\mI}{\mathbf{I}}
\newcommand{\mPhi}{\bm{\Phi}}
\newcommand{\vnul}{\bm{0}}
\newcommand{\vx}{\mathbf{x}}
\newcommand{\va}{\mathbf{a}}
\newcommand{\vw}{\mathbf{w}}
\newcommand{\vu}{\mathbf{u}}
\newcommand{\vy}{\mathbf{y}}
\newcommand{\vt}{\bm{t}}
\newcommand{\vxi}{\bm{\xi}}
\newcommand{\vmu}{\bm{\mu}}
\newcommand{\valpha}{\bm{\alpha}}
\newcommand{\vphi}{\bm{\phi}}
\newcommand{\vpi}{\bm{\pi}}
\newcommand{\vlambda}{\bm{\lambda}}
\newcommand{\Ck}{\mathcal{C}_k}
\newcommand{\mTheta}{\bm{\Theta}}
\newcommand{\mT}{\bm{T}}
\newcommand{\mW}{\bm{W}}
\newcommand{\vtheta}{\bm{\theta}}

% vim: ft=tex

\begin{document}

%-------------------------------
%   TITLE SECTION
%-------------------------------

\fancyhead[C]{}
\hrule \medskip % Upper rule
\begin{minipage}{0.295\textwidth}
    \raggedright
    \footnotesize
    Maico Timmerman \hfill\\
    10542590\hfill\\
    maico.timmerman@gmail.com
\end{minipage}
\begin{minipage}{0.4\textwidth}
    \centering
    \large
    Homework Assignment 1\\
    \normalsize
    Natural Language Processing 1, 17/18\\
\end{minipage}
\begin{minipage}{0.295\textwidth}
    \raggedleft
    \today\hfill\\
\end{minipage}
\medskip\hrule
\bigskip

%-------------------------------
%   CONTENTS
%-------------------------------

\section*{Questions 1}
\begin{enumerate}[label=(\alph*)]
    \item The general equation for the unigram-model can be written as:

        \begin{align}
            P(w_1, w_2, \ldots, w_3) &= \prodt{i=1}{N} P(w_i)\\
            &= \prodt{i=1}{N} \frac{\text{count}(w_i)}{N}\\
        \end{align}

        This is not a good solution to the \textit{their} vs. \textit{there}
        problem. Using the unigram model, the choice between \textit{their} or
        \textit{there} is based on the number of occurrences of these words in
        the training data. This approach will never take into account when a
        combination of words results in an invalid sentence.

    \item The general equation for the bi-gram model can be written as:

        \begin{align}
            P(w_1, w_2, \ldots, w_3) &= P(w_1)\cdot\prodt{i=1}{N-1} P(w_{i+1}
            \mid w_i)\\
        \end{align}

        This model will take into consideration that some occurrences of
        their/there might not be valid after certain words. Based on the
        statistics in the training set, the model will decide which of the two
        words is valid after, the seen word. However, it is still possible that
        the model will not choose the correct form.

\end{enumerate}
\section*{Question 2}

\begin{enumerate}
    \item
        \begin{align}
            \text{I have a dog that walks on the grass.}
        \end{align}

        Here the independence assumption is violated, because it could is
        possible that:

        \begin{align}
            P(\text{walks} \mid \text{a}, \text{dog}) >
            P(\text{that} \mid \text{a}, \text{dog})
        \end{align}

        However that would result into the sentence: ``I have a dog
        walks\ldots'', resulting in an invalid sentence. Using the same
        probabilities also a valid sentence can be constructed: ``A dog walks on
        the grass''.

    \item
        \begin{align}
            \text{Computers produced in a factory make mistakes too.}
        \end{align}

        violates the independence assumption, as ``make'' assumes that Computers
        is plural. Inserting a singular noun, will result in an invalid
        sentence: ``A computer produced in a factory make mistakes too.''. In
        this case ``make'' should have been ``makes'', due to ``A computer''
        being singular.

    \item
        \begin{align}
            \text{He told him something about himself.}
        \end{align}

        violates the independence assumption, as the pronoun ``himself'' is
        depended on ``He''. Changing ``He'' to ``She'' will render an invalid
        sentence, as ``himself'' should then become ``herself''.
\end{enumerate}

\section*{Question 2}
\begin{enumerate}
    \item The transition probability matrix for the corpus is:

        \begin{tabular}{c|ccccc}
            & <s> & OTH & PER & ORG & </s>\\\hline
            <s>& 0&$\frac{3}{5}$&$\frac{2}{5}$&0&0\\
            OTH&
            0&$\frac{75}{88}$&$\frac{1}{88}$&$\frac{7}{88}$&$\frac{5}{88}$\\
            PER& 0&$\frac{1}{2}$&$\frac{1}{2}$&0&0\\
            ORG& 0&$\frac{7}{16}$&0&$\frac{9}{16}$&0\\
            </s>&0&0&0&0&0\\
        \end{tabular}

    \item

        \begin{align}
            P(\text{Obama}\mid\text{T}) &= \frac{C(\text{Obama} \mid
            \text{T}) + 1}{C(\text{T}) + V_{\text{T}}}
        \end{align}

        Where we define $V_{\text{T}}$ as the subset of the vocabulary that has
        the T-tag, $P(C(\text{Obama}\mid\text{T}))$ the number of times
        ``Obama'' is tagged with the T-tag, and $C(\text{T})$ is the number of
        words tagged with the T-tag.

        Using this we calculate  $P(\text{Obama}\mid\text{PER})$ and $P(\text{Obama}\mid\text{ORG})$:

        \begin{align}
            P(\text{Obama}\mid\text{PER}) &=\frac{3+1}{6+4} = \frac{2}{5}
        \end{align}


        \begin{align}
            P(\text{Obama}\mid\text{ORG}) &=\frac{1}{16+9} = \frac{1}{25}
        \end{align}

    \item Context will not be able to disambiguate between the LOC- and the
        ORG-tag. Especially since we are only using limited set of four tags, we
        can not disambiguate the named entity using the preceding words, as they
        all have the OTH-tag.

        For example ``at the University of Chicago Law School'' can be
        interpreted as being handled by the organisation or be interpreted as
        located at the university that is in Chicago.

    \item Other sources of information that might help an automatic NER system
        could be:

        \begin{description}
            \item[Capitilization] Words containing capital letter at the start
                or in the middle are more likely to be named entities.
            \item[Hyphenation] Words that are hyphenated are more likely to be a
                named entity.
            \item[Digits] words containing digits can often be labeled as named
                entities.
            \item[Word length] Lengthy words could be an indication for named
                entities.
            \item[Punctuation] Named entities like acronyms often contain dots.
        \end{description}

\end{enumerate}
\end{document}
